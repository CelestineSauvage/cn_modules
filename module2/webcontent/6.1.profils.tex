\documentclass[12pt]{article}

\usepackage[french]{babel}
\usepackage[utf8]{inputenc}
\usepackage{geometry}
\title{Le profil des internautes}
\author {Culture numérique -- Lille 3}
\date{}
\begin{document}

\maketitle

Votre âge, votre adresse,   vos achats
récents, vos goûts musicaux, vos films préférés, vos amis, etc,
toutes ces données peuvent intéresser de nombreuses sociétés et
organisations soit pour vous surveiller soit pour vous vendre quelque
chose. Rassemblées, elles contribuent à définir votre \og
profil\fg{}. 

\section*{Profils et cookies}


Grâce aux cookies contenant des numéros d'identification, des sites ou
des jeux, sur PC, en ligne ou sur smartphone peuvent contribuer à
créer et compléter nos profils. Souvent c'est même à notre insu, en
mémorisant nos parcours sur le site, les pages visitées, etc.. Cette
collecte peut même être assurée par le biais de sites partenaires grâce à la
technique des cookies tiers.


\section*{Les réseaux sociaux - pistage systématique}

Les réseaux sociaux sont parmi les plus grands adeptes de la création
de profils. Bien évidemment de nombreuses informations personnelles
s'y trouvent, directement données par l'utilisateur, vous-même. Mais la collecte
s'étend même au delà des pages du  réseau social lui-même.

Les petits boutons \og j'aime\fg{}, \og G+\fg{} et autre \og tweeter\fg{} qui proposent de
nous faciliter le partage sont en fait des mouchards très
puissants. Présents sur une multitude de sites, ce sont des ressources
tierces, provenant des serveurs des réseaux sociaux eux-mêmes.

En effet, les boutons cachent souvent des petits programmes appelés
scripts qui informent systématiquement Facebook ou Google de votre
passage sur les sites où le bouton est présent, même si vous ne
cliquez pas dessus, ...

Dès que vous affichez sur votre navigateur une page n'ayant pourtant
rien à voir avec Facebook ou Google mais contenant l'un de ces boutons
de réseau social, le script associé envoie toutes les informations
disponibles au serveur (l'ip, le type de navigateur, ...  et surtout
le site consulté).

En plus, même si  vous n'êtes pas à ce moment là connecté à Facebook,
ou même si vous n'êtes pas membre de ce réseau, toutes ces
informations sont associées à votre profil. Ainsi même si vous ne
\og likez\fg{} pas de pages, Facebook et Google savent beaucoup de choses sur
votre navigation et vos habitudes. Votre profil prend alors de la
valeur sur le marché publicitaire.

\section*{Nos profils mis aux enchères}
Enfin, pour conclure, nous allons expliquer comment nous sommes mis
aux enchères en permanence. La plupart des sites commerciaux qui
affichent de la publicité travaillent avec des régies
publicitaires. Ces régies publicitaires travaillent elles-mêmes avec
une multitude d'annonceurs.

À chaque fois qu'un espace de publicité est disponible dans une page, la
régie soumet à ses  différents clients (les annonceurs donc) le profil
de l'internaute. En fonction des caractéristiques du profil, les
annonceurs sont prêts à payer plus ou moins cher cet espace. La
régie organise donc une vente aux enchères de notre profil. Le plus
généreux remporte le droit d'afficher sa publicité sur notre écran. 

Tout cela se déroule de manière automatique grâce à des algorithmes
sophistiqués en quelques fractions de seconde.

Ainsi la page qui héberge la publicité est payée par un annonceur qui
a choisi le meilleur prix pour son annonce et la régie prend son
pourcentage au passage. Le web est envahi par ce système complexe mais
très efficace. C'est ce qui explique que n'avons pas tous les mêmes
publicités qui s'affichent pour une même page.

\section*{La minute citoyenne}
Le web est une formidable source d'informations, un lieu d'échanges,
qui regroupe un ensemble d'outils très performants et utiles. C'est
aussi un facteur de développement économique. Mais nous l'avons
illustré, c'est également un moyen de surveillance pour les états, les
entreprises. C'est un facteur de dissémination de notre vie privée et
de collecte d'information à notre sujet, parfois,... souvent, à notre
insu. 

Vous avez maintenant les clés pour comprendre ces questions. Vous
pouvez en toute connaissance de cause, et c'est bien le droit de
chacun, laisser faire les mouchards, les régies publicitaires et tous
les collecteurs d'informations privées.

En revanche, si vous considérez que vos données vous appartiennent et
que vous n'avez pas envie d'être pisté ni ciblé, alors vous pouvez
utiliser les connaissances vues dans ce cours pour paramétrer votre
navigateur et avoir des stratégies qui visent à vous protéger. Vous
pouvez interdire systématiquement tous les cookies sur votre
navigateur, mais dans ce cas, très peu de sites continueront à
fonctionner correctement, car les cookies sont aussi utiles. Mais
votre navigateur permet un paramétrage plus fin. Vous pouvez étudier
ces paramètres et  par exemple :

\begin{itemize}
\item interdire les cookies tiers (ils sont souvent autorisés par
  défaut),
\item limiter la conservation des cookies et même les effacer
  régulièrement
\end{itemize}

Vous pouvez également installer des modules complémentaires bloquant
les publicités, les boutons de réseaux sociaux, ou les mouchards en
tout genre, ...

Enfin, si vous pensez que vos droits de citoyens sont bafoués sur le
web, c'est sûrement sur le plan juridique que la bataille doit avoir
lieu. Vous êtes maintenant mieux armés pour rejoindre les différentes
associations d'utilisateurs, ou pour interpeller les élus, participer
aux débats publics sur les questions de respect de la vie privée. 

\end{document}

%%% Local Variables:
%%% mode: latex
%%% TeX-master: t
%%% End:
