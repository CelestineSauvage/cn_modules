\documentclass[12pt]{article}

\usepackage[french]{babel}
\usepackage[utf8]{inputenc}
\usepackage{geometry}
\usepackage[hidelinks]{hyperref}
\title{Les moteurs de recherche}
\author {Culture numérique -- Lille 3}
\date{}
\begin{document}

\maketitle

\section*{Des ressources qui n'existent que quand on les demande\ldots{}}
\label{sec:orgheadline1}
Prenons l'exemple de l'URL suivante :

\begin{center}
  {\url{http://www.univ-lille3.fr/etudes/orientation-emploi/}.}
\end{center}

 Rappelons que la
partie \texttt{etudes/orientation-emploi} désigne une ressource sur le
serveur web \texttt{www.univ-lille3.fr}.  Il est possible que ce soit un
document composé par une personne du service des études puis
enregistré sur les disques durs de ce serveur web pour le mettre à
disposition des internautes. Mais à vrai dire, c'est un processus de
conception à la mise en ligne de ressources aujourd'hui de plus en
plus rare.  Dans le web moderne, de plus en plus souvent, ces
ressources sont composées par des programmes informatiques, à partir
d'éléments pris dans de nombreuses sources de données. Ces programmes
sont par exemple des outils de publication web, systèmes de gestion de
contenu (CMS en anglais), des wiki, des moteurs de blogs\ldots{}

Mais un autre exemple évident de la génération automatique de
ressources est celui des moteurs de recherche. Lorsque vous appuyez
sur le bouton de recherche après avoir saisi vos mots clefs, le
document qui apparaît dans votre navigateur a évidemment été construit
juste pour vous, au moment de votre demande.

\section*{Un annuaire de toutes les ressources}
\label{sec:orgheadline2}
Le web est un immense ensemble de ressources reliées entre
elles. On pouvait imaginer à ses débuts parcourir cet ensemble et
trouver son chemin vers la ressource souhaitée. On a donc commencé à
construire des annuaires et des répertoires à l'image de ce qui peut
se faire dans des bibliothèques. Tim Berners Lee, inventeur du web, a
même maintenu une liste de serveurs web à cette époque. Mais cet idéal
a rapidement été abandonné.  La taille du web a grandi tellement vite
qu'il est devenu impossible de consigner les adresses de toutes les
ressources, ou même seulement les plus importantes. C'est alors que
sont entrés en jeu les moteurs de recherche.

\section*{Comment fonctionne un moteur de recherche aujourd'hui}
\label{sec:orgheadline3}
Comment fonctionne un moteur de recherche ? C'est à la fois simple
dans certains principes généraux et complexe pour de nombreux détails
importants. C'est à la fois connu dans sa généralité et bien caché
dans ses détails. Nous nous contentons ici de simples généralités.

Les moteurs de recherche construisent constamment, car le web évolue
sans cesse, un index. L'index, c'est comme dans un livre, un moyen
d'aller directement à une page à partir d'un mot. Pour construire un
tel index, il faut avoir lu toutes les pages du livre et consigné pour
tous les mots, la liste des pages où ils se trouvent. Les moteurs de
recherche téléchargent toutes les ressources du web en permanence pour
extraire la liste des mots qu'on y trouve et garder l'énorme liste des
URLs où ces mots se trouvent. Ce ne sont pas des hommes qui parcourent
le web pour eux, mais des programmes, appelés des robots. Les robots
sont les clients des serveurs web les plus nombreux et réguliers\ldots{} et
de loin!

Mais afficher simplement la liste de ces ressources quand l'internaute
saisit quelques mots dans le formulaire de recherche n'est pas
satisfaisant. La liste est bien trop longue. Le deuxième ingrédient du
moteur de recherche est le programme qui permet d'interroger cet
index, simplement en lui donnant quelques mots, et qui construit une
liste, présentée par ordre d'importance, d'URLs désignant les
ressources où ces mots se trouvent.


La magie des moteurs de recherche tient dans les détails qui
permettent à l'ensemble de fonctionner tels que l'existence d'un index
à jour, la forme de l'index qui permet d'y retrouver extrêmement
rapidement les pages associées à un mot, ou encore l'ordre
d'importance dans lequel les résultats de l'interrogation de l'index
apparaissent.

L'avance technologique des grands moteurs de recherche se cache dans
les détails de la construction de l'index mais surtout du programme
qui permet de l'interroger et de la détermination de l'ordre des URLs
affichées en retour. Ces détails sont protégés par de nombreux secrets
industriels.

\section*{Collecte de données d'usage}
\label{sec:orgheadline4}

Mais un avantage qui rend la mise en concurrence des grands moteurs de
recherche actuels presque impossible tient à un dernier
paramètre. C'est la disponibilité d'énormes quantités de données
d'usage, parfois personnalisées. En effet le résultat (l'ordre
d'apparition des ressources) des requêtes au moteur dépend aujourd'hui
fortement de ce qu'ont fait leurs utilisateurs : sur quels liens
ont-ils cliqué ? À l'inverse des ressources du Web derrière les URLs,
ces données d'usage ne sont pas publiques, mais sont tout aussi
cruciales pour générer des réponses aux requêtes dans un ordre pertinent.

En conséquence, les moteurs de recherche collectent sans cesse des
données à propos de vos recherche. La tendance actuelle est de rendre
les réponses personnalisées, ce qui entraîne une collecte de données
personnelles rendue possible à la fois par les techniques de cookies
et l'utilisation de comptes chez ces opérateurs de recherche.

\section*{Modèle économique du moteur de recherche}
\label{sec:orgheadline5}

Pour une institution qui veut être visible sur internet, if faut
assurer sa présence dans l'index. Mais cela n'est pas suffisant : il
faut être en haut de la liste et donc apparaître important aux yeux du
moteur de recherche.

De bonnes pratiques en matière de conception de pages web peut y
contribuer. Puisque toute la chaîne de traitement est automatique, les
ressources que le moteur analyse et indexe doivent être parfaitement
intelligibles par la machine. Il est donc très important d'écrire
correctement ses pages web dans ce but de traitement automatisé autant
que dans le but de se faire comprendre de ses lecteurs
humains. Parfois des conseillers un peu charlatans tentent de se faire
passer pour des gourous qui vont propulser des sites en première page
des résultats de recherche.

Il faut s'en méfier car pour le moteur de recherche, une des premières
sources de revenu est de vendre ces places. Cela se traduit
littéralement par des \emph{ventes de mots}. Une deuxième source de revenu
est liée à la collecte des données personnelles des
utilisateurs. Tirer des informations à l'insu ou non de ses usagers
n'est pas une pratique réservée aux moteurs de recherche. De nombreux
autres acteurs du web fonctionnent sur ce même principe.

\section*{Aller plus loin}
\label{sec:section}

Cette petite introduction des moteurs de recherche est volontairement
très succinte et parcellaire. Des éléments techniques essentiels ne
sont pas mentionnés comme :
\begin{itemize}
\item les pré-traitements des textes et la sélection du vocabulaire,
  le traitement des majuscules, des accents etc...
\item le calcul du score de pertinence sur lequel repose cet ordre
  d'affichage des réponses, et bien-sûr
\item l'un des algorithmes les plus connus qu'est PageRank utilisé par
  Google.
\end{itemize}

Nous vous invitons à suivre les cours d'option transversale en
licence, les options de master sur les humanités numériques, ou les
prochains cours de culture numérique qui aborderont sans doute ces
questions beaucoup plus précisément.


\end{document}
%%% Local Variables:
%%% mode: latex
%%% TeX-master: t
%%% End:
